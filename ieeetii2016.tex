
%% bare_jrnl.tex
%% V1.4b
%% 2015/08/26

\documentclass[journal]{IEEEtran}

\usepackage{cite}
\usepackage[hyphens]{url}
\usepackage[pdftex]{graphicx}
\usepackage{paralist}
\usepackage[pdftex,colorlinks=true]{hyperref}

% correct bad hyphenation here
%\hyphenation{op-tical net-works semi-conduc-tor}


\begin{document}
%
% paper title
% Titles are generally capitalized except for words such as a, an, and, as,
% at, but, by, for, in, nor, of, on, or, the, to and up, which are usually
% not capitalized unless they are the first or last word of the title.
% Linebreaks \\ can be used within to get better formatting as desired.
% Do not put math or special symbols in the title.
\title{Digitally-Enabled, As-a-Service, Electric Vehicles: Exploring the `TriOpt' of Private Transport}
%
%
% author names and IEEE memberships
% note positions of commas and nonbreaking spaces ( ~ ) LaTeX will not break
% a structure at a ~ so this keeps an author's name from being broken across
% two lines.
% use \thanks{} to gain access to the first footnote area
% a separate \thanks must be used for each paragraph as LaTeX2e's \thanks
% was not built to handle multiple paragraphs
%

\author{Peter~Cooper, Theo~Tryfonas and~Tom~Crick}% <-this % stops a space
%\thanks{}% <-this % stops a space
% \thanks{M. Shell was with the Department
% of Electrical and Computer Engineering, Georgia Institute of Technology, Atlanta,
% GA, 30332 USA e-mail: (see http://www.michaelshell.org/contact.html).}% <-this % stops a space
%\thanks{J. Doe and J. Doe are with Anonymous University.}% <-this % stops a space
%\thanks{Manuscript received April 19, 2005; revised August 26, 2015.}}

% note the % following the last \IEEEmembership and also \thanks - 
% these prevent an unwanted space from occurring between the last author name
% and the end of the author line. i.e., if you had this:
% 
% \author{....lastname \thanks{...} \thanks{...} }
%                     ^------------^------------^----Do not want these spaces!
%
% a space would be appended to the last name and could cause every name on that
% line to be shifted left slightly. This is one of those "LaTeX things". For
% instance, "\textbf{A} \textbf{B}" will typeset as "A B" not "AB". To get
% "AB" then you have to do: "\textbf{A}\textbf{B}"
% \thanks is no different in this regard, so shield the last } of each \thanks
% that ends a line with a % and do not let a space in before the next \thanks.
% Spaces after \IEEEmembership other than the last one are OK (and needed) as
% you are supposed to have spaces between the names. For what it is worth,
% this is a minor point as most people would not even notice if the said evil
% space somehow managed to creep in.



% The paper headers
%\markboth{Journal of \LaTeX\ Class Files,~Vol.~14, No.~8, August~2015}%
\markboth{IEEE Transactions on Industrial Informatics, February~2017}%
{Shell \MakeLowercase{\textit{Cooper et al.}}: Digitally-enabled,
  as-a-Service, Electric Vehicles: Exploring the `TriOpt' of Private
  Transport}
% The only time the second header will appear is for the odd numbered pages
% after the title page when using the twoside option.
% 
% *** Note that you probably will NOT want to include the author's ***
% *** name in the headers of peer review papers.                   ***
% You can use \ifCLASSOPTIONpeerreview for conditional compilation here if
% you desire.


% If you want to put a publisher's ID mark on the page you can do it like
% this:
%\IEEEpubid{0000--0000/00\$00.00~\copyright~2015 IEEE}
% Remember, if you use this you must call \IEEEpubidadjcol in the second
% column for its text to clear the IEEEpubid mark.

% use for special paper notices
%\IEEEspecialpapernotice{(Invited Paper)}

% make the title area
\maketitle

% As a general rule, do not put math, special symbols or citations
% in the abstract or keywords.
\begin{abstract}
Three distinct trends are emerging to shake-up the dominance of
privately-owned, combustion car transport in the UK. First is the
emergence of the electric powertrain as an affordable means of
transport. This carries the potential to address many of the
‘pump-to-tire’ shortcomings, especially CO2 emissions, air and noise
pollution. The second is the rise of new hire models of car ownership
-- the concept of paying for the use of a car per single leg of a
journey, also addressing residential parking and social
division. Thirdly, the rise of `smart city thinking', the concept that
increased connectivity and data can be harnessed to create value,
including in transport. We define the combination of the three as the
`TriOpt' of private transport – three disruptors that should not be
considered in isolation but as interacting levers -- an inflection of
the `Energy Trilemma'.

In this paper, we apply systems-thinking and utilise a mixed
methodology of workshops, interviews and systems modelling drawn upon
the UK city of Bristol's `Smart EV Transport Hub' project. The
objectives are to identify concepts that positively combine two or
more of these three `Opts' and understand the barriers and enablers to
these. Subsequently, the use cases are evaluated qualitatively for
their comparative value, and segmentation is undertaken to
characterise and generalise groups of concepts to inform recommended
stakeholder actions. We demonstrate that synergistic overlaps are many
and they create significant value, thus this `TriOpt' notion should be
investigated further. Our data highlights that of the greatest value
are those use cases that the current literature base has explored the
least, and can be characterised as requiring significant public and
private sector collaboration. It is strongly recommended that
public-private sector collaboration in private transport, within a
context of digital innovation, electric vehicles and mobility as a
service model, is subject to further investigation.
\end{abstract}

% Note that keywords are not normally used for peerreview papers.
% \begin{IEEEkeywords}
% Electric Vehicles, Vehicle Hire Models, Smart Technologies, Smart
% Monitoring, Smart Cities, Big Data, Environmental Impact, Business
% Models, Mobility-as-a-Service 
% \end{IEEEkeywords}


% For peer review papers, you can put extra information on the cover
% page as needed:
% \ifCLASSOPTIONpeerreview
% \begin{center} \bfseries EDICS Category: 3-BBND \end{center}
% \fi
%
% For peerreview papers, this IEEEtran command inserts a page break and
% creates the second title. It will be ignored for other modes.
\IEEEpeerreviewmaketitle



\section{Introduction}
% The very first letter is a 2 line initial drop letter followed
% by the rest of the first word in caps.
% 
% form to use if the first word consists of a single letter:
% \IEEEPARstart{A}{demo} file is ....
% 
% form to use if you need the single drop letter followed by
% normal text (unknown if ever used by the IEEE):
% \IEEEPARstart{A}{}demo file is ....
% 
% Some journals put the first two words in caps:
% \IEEEPARstart{T}{his demo} file is ....
% 
% Here we have the typical use of a "T" for an initial drop letter
% and "HIS" in caps to complete the first word.
% \IEEEPARstart{T}{his} demo file is intended to serve as a ``starter file''
% for IEEE journal papers produced under \LaTeX\ using
% IEEEtran.cls version 1.8b and later.
% You must have at least 2 lines in the paragraph with the drop letter
% (should never be an issue)

\subsection{Problem Space}

\IEEEPARstart{A}{growing} literature base is gathering consensus on
the view that the current Western private transport paradigm has a
finite lifespan; a transport culture that consists of overwhelmingly
privately-owned internal combustion engine automobiles is unlikely to
survive the next 50 years in its current form, in the face of
economic, social, and environmental
pressures~\cite{lerner:2011,parkhurst:2011,van-audenhove-et-al:2014,black-et-al:2016}.
Several distinct trends have emerged a as potential disruptors in
existing research. Three in particular are identified and focused upon
within this paper.  Firstly, the emergence of electrical motors as the
primary alternate powertrain for private
automobiles~\cite{paffumi-et-al:2015,gnann-et-al:2015}.  Secondly --
and more in its infancy -- the trend of transitions to new car use
models, which is regularly referred to within the collective term of
`Mobility as a Service'~\cite{tscatapult:2016}. A third -- the
broadest and most in its infancy -- the emergence of the smart city
theorem. This capitalizes on the use of systems that harness the
opportunities increased data and connectivity present to provide value
(Townsend, 2013) (IBM, 2014) where disruption to private transport,
measured on any one of these individual trends is, at present,
slow. Electric cars only make up a negligible share of the UK car
market (with limited charging infrastructure outside of major urban
areas) [2]; short-term hire transport models have yet to be proved at
a significant scale in the UK, beyond simpler transport situations,
such as bicycles[REF3]; and smart cities are, in many cases, little
more than a long term strategic aspiration, with some instances of
demonstrators, most of which are too early in their lifetime to be
able to provide any substantial conclusions about the value of data.

Some of the most successful disruptive private transport initiatives
of recent times can however often be observed combining two or three
of these opportunities. AutoLib, for example is one such case
study[REF4]. The Paris-based EV car hire scheme offers one-directional
trips around the city and since 2011 has already grown to over 500,000
members and 4,000 cars. Tesla Motors have also heavily emphasized new
business models enabled by data and the role of new ownership models
in their latest corporate strategy release (Musk, 2016). Existing
literature has extensively examined each trend in its isolation, and
to a limited degree there is exploration of combinations of two
(Graham Parkhurst, (in draft)), but there is almost no consideration
of triple-overlaps (see Figure~\ref{fig:triopt}), nor of the notion of
synergy between the three trends as a principal. The idea of three
significant issues needing to be considered in conjunction with one
another is not a radical concept however, as seen in the inverse, but
principally similar, `energy trilemma' [REF4a].  This paper --
building upon previous work in this
space~\cite{cooper-et-al-sose:2015} -- will focus on investigating the
manifestations of the overlaps of these three opportunities within the
context of the city of Bristol in the UK.

\begin{figure}[!htb]
\centering
\includegraphics[width=\columnwidth]{images/triopt.png}
\caption{The proposed `Tri-Opt' of positive opportunities to disrupt
  the private sector transport paradigm in developed countries. Areas
  of double overlap correspond to a perception of the strengths and
  weaknesses in these combinations, as presented in the literature
  reviewed; the area of triple overlap is the stated hypothesis for
  this research.}
\label{fig:triopt}
\end{figure}


\subsection{Electric Vehicles}

Electric vehicles (EVs) can offer an environmentally sustainable
alternative to internal combustion engines (ICEs). EVs are powered by
a battery which is charged through the electricity network. Alongside
reducing carbon emissions, EVs can also improve noise and air quality,
provide savings for consumers and reduce dependency on specific fossil
fuels [1]. It is widely accepted that an alternate energy source is
necessary for the UK transport network in the near future, and
electrification is currently considered the most likely choice. The
Department for Transport predicts that by 2020 there will be 1.5m EVs
on the road in the UK [2].

Most of the world's major automotive companies have released
purpose-designed electric cars (as a opposed to `engine-swap-out'
models); some have gone so far as to make significant strategic
investment in the concept by releasing an entire electric car range,
for example BMW's i Series range. The direct `pump-to-tire' environmental
benefits of electric cars, lower particulate emissions, lower noise
emissions and lower operational CO2 emissions, are highly
significant. Legislation in many countries is acting in two ways:
penalizing internal combustion engine users, and incentivizing the
purchase of EVs. Electric cars bring with them several caveats,
however: the capital cost of EVs, predominantly due to current battery
technology, is yet to be comparable to an equivalent internal
combustion engine car; the embodied carbon of EVs, again due to the
battery component, is on average considerably higher than an
equivalent internal combustion engine, and the generation of
electricity to meet charging patterns is theorized to bring with it
considerable logistical difficulties on national grids [REF5].

There are a number of barriers to the adoption of EVs. Many of these
are psychological, for example range anxiety; consumers worry that
they may `run out of juice' [3]. However, 95\% of all vehicle journeys
in the UK are less than 25 miles (40km) [3], only a small proportion
of existing EV's range. Other consumer barriers include: concerns over
battery lifetimes, the risk associated with investing in a relatively
new technology and the current, comparatively large capital investment
required to purchase an EV.  Personal vehicles and small commercial
vehicles account for 13\% of all UK carbon emissions [4]; by
transitioning to EVs in these sectors, the overall volume of emissions
could be significantly reduced. Due to the nature of UK car culture,
fleet vehicles accounted for 63\% of all new vehicle sales in the UK
in 2011 and as such are a dominant influence on the type of cars for
sale in the used market [5]. There is a growing trend for EVs in fleet
vehicles, so it is likely a tangible used EV market will start to
emerge in the next five years.


\subsection{Digital Innovation and Smart Technologies}

Smart technology is a new and rapidly growing concept; as such, a
consensus of its exact definition has yet to be reached. However,
interviewed academics [7], industrial experts [8] and a number of
papers [9]–[12] identify the fundamental, unifying theme as the use of
data and connectivity to produce value. The rise of interest in
data-based possibilities for the built environment is regularly cited
as being fuelled by three key developments:

\begin{enumerate}
\item {\textbf{The rapid increase in the production of data.}}
Computer scientists have stated that many social trends, such as the
rise of internet connectivity (specifically high-speed mobile
Internet), and social networking, has caused an exponential increase
in the production of data. Today, almost 30 petabytes of data exists
on Facebook
alone\footnote{\url{http://newsroom.fb.com/company-info/}}. The
abundance of such a resource has spurred consideration as to its
potential use.
\item {\textbf{The rapid increase in the ability to collect specific
data.}} Improvements in sensor and communication technology have meant
the installation of data collection devices is now both financially
and spatially practical (Townsend, 2013). The development of mesh
networks, the notion of deploying distributed sensors across a large
area, can provide a high resolution or accuracy of data or the ability
to track the movement of entities. Furthermore, the Internet of
Things, the notion that with the deployment of connected sensors on
existing everyday objects, interactions that are currently
machine-human- machine could simply be machine-machine, allowing many
such processes to be faster, cheaper and more convenient.
\item {\textbf{Improvements in data storage and processing.}}
Following Moore's Law/Kryder's Law, storage is far cheaper than ever
before, allowing this vast data to be stored. Processing power is also
much greater, allowing complex trends from data on scales so vast as a
city to be processed and actioned fast enough to be considered `live'.
\end{enumerate}

Many such sources refer to data as a raw material (perhaps even an
emerging `utility' (Socrata, 2015), joining electricity, gas, water
and telephone networks), creating the notion that data can and should
be used as an input to a business model that is then translated to
value. In recent years the practicalities of collecting and processing
a vast quantity of high value data of a system, as discussed, has
developed significantly [10]. Such a system could be a house, business
or city, with data being the movement of containers in a factory, the
journeys of cars across an urban network, or the use of electricity
residentially. The sheer volume of data available for such a large
range of systems has led to the term `big data' now commonly used to
refer to a datastream large enough to make smart value-producing
systems feasible [13]–[15]. This may involve taking data across
traditional `silos' [17] [18]. `Creating value' can be interpreted in
many ways, including the translation of a process to be:

\begin{itemize}
\item {\textbf{Faster:}} for example, using traffic data to update
  signs in real time, rather than the slow, reactive methods by which
  traffic is informally advised against taking certain routes.
\item {\textbf{Fairer:}} for example, demand-based pricing for grid
  electricity, whereby data from the national grid is used to charge
  those who use electricity when demand is highest (when the cost of
  generation is highest), a cost that is reflective of the cost to
  supply the electricity to them, and vice versa in periods of low
  demand.
\item {\textbf{At lower cost:}} for example, flow sensors in water
  pipes can be used to deduce the exact location of leaks when they
  emerge, rather than expensive and time-consuming visual
  inspections. [16]
\item {\textbf{Without human interaction:}} for example, when an
  inspection robot is automatically sent to the site of a machine
  malfunction in a factory, rather than a human noticing the fault and
  piloting the robot manually. This can improve instances of both
  human error, hesitation and subjective judgment (for better or for
  worse).
\end{itemize}

While there is acknowledgement that it is not without ethical and
societal concerns (Bimber, 1990), there is an extensive literature
base documenting the potential impact of digital innovation, through
these improvements, in the transport sector (Enoch, 2015).


\subsection{New Ownership Models}

`Mobility as a Service' has grown to be a concept that is specifically
appreciated in modern transport dialogues (Transport Systems
Catapault, 2016). It is best defined as a transition from a paradigm
of relying on a product that is purchased to provide mobility
functionality, to a service where the outcome of moving from one
location to another is provided, disassociated from any requirement
for asset ownership, and transitionally arranged on a
journey-by-journey basis.

In other modes of private transport in the UK, particularly bike use,
an increasing number of users are opting to participate in short term
hire models of use, particularly in urban contexts (Meddin,
16). Rather than bearing the capital and logistical cost of owning a
bike, individuals hire the bike for a nominal fee from a given node
near their origin, complete their journey, and return the bike to a
node near their destination. In essence, they are purchasing mobility
per-leg of their journey, or `as-a-service'. Once seen as radical,
examples such as London’s cycle hire (``Boris Bike'')
scheme\footnote{\url{https://tfl.gov.uk/modes/cycling/santander-cycles}}
demonstrated not only the feasibility of the business model, but also
the efficacy of the indirect benefits, illustrated by significant
increase in cycling in the city bringing health benefits, but also the
notion of these as a substitute for car journeys.

A range of drivers have been suggested for the emergence of this paradigm:

\begin{itemize}
\item {\textbf{Changing societal values:}} whereby evidence has
suggested a decrease in the value placed on car ownership culturally
(Rosenthal, 2013); conflict with an increased desire to live in
vibrant urban areas within walking distance of workplace and other
amenities and the spatial restrictions of car ownership in such
scenarios (Michael Jenks, 2000).
\item {\textbf{Changing economic situations:}} increasing costs of car
ownership, particularly in insurance (particularly for young
individuals) and fuel cost. (MultiCar, 2015) (Blackmore, 2015)
\item {\textbf{Changing effectiveness of privately-owned car
transportation:}} an increasing frustration with congested transport
systems and an increasing desire to travel to A-to-B reliably,
regardless of the specific comfort of one's `own' motor vehicle.
\item {\textbf{Proof of concept:}} Driven by commercial ventures
showing the viability of alternate private transport
paradigms. Traditional car hire companies in particular are beginning
to explore short-term, distributed `car club', return journey
(`A-to-A' journeys) offerings, whereas emerging start-ups are offering
entire by-leg services, such as the aforementioned Autolib. (Wefering,
2016)
\end{itemize}

% \subsection{The UK Context}

% In the UK, over the last 60 years, the profile of citizen journeys has
% transitioned rapidly from public transport dominated to private
% transport-dominated, stabilizing around the last millennia
% [FOOTNOTE1]. Vocal, anti-automotive political agendas, global
% recession and rising oil prices have all emerged since as powerful,
% anti-private transport influences. However, they have done little to
% reduce private transport, with the per-mile travel profile appearing
% largely unchanged, tending instead to a marginally lower equilibrium
% level from 2000 to today. This is displayed in Figure 1a,
% contextualized by the historical car ownership trends in Figure
% 1b. However, there is emerging evidence that over the last 10 years,
% cars per person are declining, driven by lower ownership rates,
% particularly driven by younger drivers [REF6]. 


% \begin{figure}[!htp]
% \centering
% \includegraphics[width=\columnwidth]{images/uktransportshare.png}
% \caption{Car ownership trends in the UK}
% \label{fig:uktransportshare}
% \end{figure}

% \begin{figure}[!htp]
% \centering
% \includegraphics[width=\columnwidth]{images/ukcarsperperson.png}
% \caption{Transport share in the UK}
% \label{fig:ukcarsperperson}
% \end{figure}

% In theory, it is clear that at present public transport is both
% technically (in terms of emissions per head) and systematically
% (second-order effects, such as congestion, social and economic
% consequences) superior to private transport and this has been
% understood for some time [REF7]. However, it is equally apparent that
% the unit cost of converting a percentage of transport users from
% private to public transport increases with conversion. In other words,
% in a hypothetical world where everyone uses private transport, to
% convert the easiest first 10\% of users, a network of urban buses and
% trains can be run cost-effectively and efficiently. The last 10\% of
% the population, however, requires a vast and complex network that
% operates constantly and at high frequencies, and is for all intents
% and purposes, impossible to deliver. As such, it is clear that there
% is a fraction of transport that, for reasons of financial and
% practical limitations, will always remain private transport, and that
% that fraction is not necessarily `small'. Although the UK does not
% currently operate at or even near this ‘minimum’ private transport
% threshold, acknowledgement that any such unavoidable level exists is
% not necessarily popular in transport policy.

% As such, there is clear motivation that the impacts of private
% transport, both pump-to-tire and systematically, should be
% mitigated. This is a popular and high-growth area of research both
% commercially and academically; within the last 20 years, the average
% personal automobile efficiency has improved drastically.  Today,
% alternate  powertrains  such as electric motors that are more
% efficient and less polluting that internal combustion engines, are not
% only technically viable, but a key component of many automotive giants
% strategies; the aforementioned BMW i Series range is a high profile
% example of significant strategic investment in the transition of
% retail automobiles to EVs. However, it is clear that drivetrain
% transitions alone cannot deal with the systematic drawbacks of private
% transport, such as congestion, spatial requirements of parking and
% social isolation of the have-cars and have-nots, which have been
% addressed to an order-of-magnitude lesser extent. Technical
% suggestions for these problems do exist, such as driverless, automated
% cars and personal rapid transit (PRT) schemes, but both are a long way
% from a ready-to-rollout status [REF8]. Crucially, for every change
% that reduces the negative impact of private transport (of either
% kind), or more accurately, convincingly depicts itself as doing so,
% private transport is further legitimized, and efforts to transfer
% journeys to public transport are weakened. Thus improvements to the
% performance of private transport need to be sufficiently substantial
% to mitigate any systematic `rebound' effect of reduced public
% transport use.
 
% Private transport solutions are almost exclusively the remit of the
% private sector; if we focus on the mode of car, these can be broadly
% categorised as automotive retailers, car hire firms and taxi
% companies. The public sector’s potential influence is relatively broad
% and classifiable into two main groups. Firstly, regulation, subsidies
% and taxation, which while exercised to bring about sustainability
% improvements in car design and car use, are generally applied at an
% industry level and are balanced against other political desires such
% as keeping motor product prices low and the automotive commercial
% environment favorable to investment [REF9]. Secondly, through the
% provision of infrastructure, which while said to be becoming
% increasingly mindful of its systematic impacts, is still predominantly
% designed to meet the current and near-future needs of its users rather
% than dictate them (Institution of Civil Engineers, 2016). 

% Political action is thus limited by this simple catch-22: public
% transport is the ideal, but impractical to reach saturation. Private
% transport is undesirable, but unavoidable, yet efforts to improve it
% also make it more prevalent. Private transport services must remain
% commercially profitable if they remain in the hands of the private
% sector, and at present, the government appears not inclined to lead
% radical changes in its nature from the limited influences it has at
% its disposal. It seems change must come primarily from the private
% sector (Westlake, 2016).

% Technically, it may be argued there thus exists an optimum – a level
% of improvement of private transport that, when the reduction in public
% transport uptake is factored in, results in the lowest overall impact
% in the UK’s travel emissions. There are many more secondary effects
% that could also be considered from this first deduction; not all
% improvements to private transport reduce public transport uptake to
% the same degree; nor is the UK a homogenous mass of two types of
% transport user. Attempts to penalise private transporters who could
% transfer to public transport and do not, such as through taxation,
% frequently damages those who cannot transfer [REF10] [REF11].

% Thus we can see how the three opportunities introduced previously come
% together to address three of the main difficulties in the current
% transport paradigm in the UK:

% \begin{enumerate}
% \item {\textbf{Electric vehicles}} -- enabling significant improvement
% in the direct, pump-to-tire impacts of private transport, one of the
% most substantial steps on the long term achievements in this space;
% \item {\textbf{Mobility-as-a-Service models}} -- enabling significant
% improvement in the systematic impacts of private transport, a radical
% first step in achievement in this space;
% \item {\textbf{Digital innovation}} -- enabling significant
% improvement in operational cost, customer engagement, system
% management and new revenue streams, appeasing the private sector's
% requirement for profitable ventures.
% \end{enumerate}

% The actions of transport stakeholders in the next 10 years will
% dictate how much these trends are harnessed, encouraged or ignored in
% private sector transport, and ultimately how the UK’s transport
% culture changes as a result.


\section{Methodology}\label{methodology}

\subsection{Objectives}

This paper sets out to achieve the following objectives:

\begin{enumerate}
\item To explore a range of use cases for how the opportunities in the
  aforementioned `tri-opt' overlap in the UK's private transport, in
  combinations of two or three, to produce value.  
\item To understand, to a qualitative and relative degree of accuracy,
  the varying value perceived by our stakeholders to be presented in
  the identified use cases. 
\item To understand the broad characteristics of the identified use
  cases, with a focus on barriers and enablers to their valuation and
  implementation.
\item To segment the use cases, considering a combination of their
  characteristics and value potential, with a view to forming
  recommended policymaker actions.
\end{enumerate}

\begin{figure}[!htb]
\centering
\includegraphics[width=\columnwidth]{images/bristolhub.png}
\caption{An overview of the functionality of Bristol's Smart Electric
  Transport Hub, structured by input transport modes (left) and output
  transport services (Cooper, Elder, Noble, Tingle, \& Wilde,
  2015)}
\label{fig:bristolhub}
\end{figure}

\subsection{Philosophy}

This paper will take a systems thinking perspective to divergently
consider the instances of how these trends create double and triple
overlaps.

With respect to the definition and measurement of value, due the broad
scope of this paper, value will be considered qualitatively, with an
emphasis on a use case's relative value compared to other identified
use cases. Value will be considered to all stakeholders within the
system boundary – in this case the city of Bristol, UK.
 
To add rigor to this process, during engagement with stakeholders, the
following framework, combining three approaches for considering value,
were used:

\begin{figure}[!h]
\centering
\includegraphics[width=\columnwidth]{images/valuationmethods.png}
\caption{Valuation methods}
\label{fig:valuationmethods}
\end{figure}


\subsection{Methods}

This paper implies a mixed methods approach, primarily – but not
exclusively -- focused around a joint Bristol City Council and Bristol
University research study into the potential of a `smart electric
vehicle transport hub' -- a development that combined the three
proposed trends on a physical site offering both public (bus and `park
and ride') and private (as-a-service electric car hire) services
(Cooper, Elder, Noble, Tingle, \& Wilde, 2015).

The methods used are as follows:

\begin{itemize}
\item 5 x 2 hour workshops with the presence of senior Bristol City
Council staff, senior University of Bristol academics, and consulting
engineers from built environment consultancy Arup;
\item 10 semi structured 1 hour interviews with a range of transport
stakeholders in the city of Bristol, including bus operators, policy
legislators, legal and financial professionals, all providing insight
anonymously;
\item A survey of 48 citizens of Bristol subscribed to Source West –
an independent non-profit organization representing the interests of
citizens using electric vehicles;
\item A range systems dynamics modelling to understand quantitative
value in a few specific use cases, explained in more detail when
introduced.
\end{itemize}


\section{Results}

\subsection{Overview}

The results are presented in the following format:

\begin{itemize}
\item {\textbf{Individual Use Cases:}} A profile of each individual
use cases identified as a result of the methodology, articulated as
communally defined by stakeholders participating. Each use case is
labeled with the relevant `Opts' they involve;
\item {\textbf{Use Case Value Comparison:}} A value map of the use
cases, as described in Section~\ref{methodology};
\item {\textbf{Individual Barrier and Enablers:}} A profile of each of
the identified barriers and enablers, identified as a result of the
methodology, articulated as communally defined by stakeholders
participating. On review of the results, the cross-applicability of
these makes it logically to present these as a communal set. However
particularly strong applicability of one barrier/enabler to a
particular use case will be emphasized in that barrier/enabler's
description.
\end{itemize}


\subsection{Individual Uses Cases}

\subsubsection{Car Component State -- Smart/MaaS}

In the traditional car hire industry, it is practice is to operate a
maintenance regime that involves inspection above the recommended
frequency, designed to reduce the time between inspections when the
car may suffer from a freak failure. Freak failures are defined as
those whereby there was no indication at the last inspection that the
car would fail before the next testing.

Using health and usage monitoring sensors attached to key components
in the car, an operator of cars-as-a-service offering could gathering
insight on a car's mechanical state close to the quality of that
offered by a human inspection, in real time. This could drastically
reduce the rate of freak failures in such a service that may have
extremely fast turnarounds between regular and short customer uses. If
the sensor coverage was sufficient enough, potentially allow cut backs
in human servicing.

\subsubsection{Live Air Quality Management -- Smart/MaaS}

Bristol City Council currently monitors air quality by semi-permanent
installations at specific areas around the city. In a typical UK urban
environment, particularly one with no major industry, air quality is
primarily determined by road transport emissions. As such, if it was
possible to understand the overall distribution of vehicles in the
city at any one point in time, it is a reasonable hypothesis to
believe it is possible to estimate, with relative accuracy, the air
quality throughout the entire city. At present, in some areas of the
city, car flow is monitored by car-recognizing cameras. These however
are sparse and expensive to install.

Cars-as-a-service are likely to use the same roads to the same
intensity of other cars on the road at that time. In other words,
their road routing behavior is likely to be very similar, if not
identical, to the rest of the cars on the road at that time. As such,
a critical resolution of hire cars is necessary to deduce an estimate
of the wider car resolution. 

%We will examine critical resolutions in detail in Section~\ref{barriersenablers}.

A particular opportunity with a MaaS model is to directly affix air
quality sensors on to cars, gathering primary air quality data
directly.


\subsubsection{Live Accident Reporting -- Smart/MaaS}

Cars-as-service can also be fitted with impact sensors, altering the
MaaS management system that a car has suffered a serious crash,
allowing them to contact the authorities. Due to the significant
improvement in road safety regulations, mobile phones have already
improved road transport since the mid-2000s, and the relative rarity
of isolated crashes, it is unlikely the value case here can be made
from a practical improvement in fatality rates from car hire
use. Instead, it is probable that the main benefit of such a system
would be perceptual improved piece of mind for the customer that a
system will be in place to constantly evaluate their safety, as well
as pre-warning a traffic control system of potential disruption.


\subsubsection{User Journey Data – Smart/MaaS}

Many retail and advertisement companies using big data have
demonstrated that advertisement conversion rates (the percentage of
individuals who act on an advertisement they have seen) can be greatly
increased by accurate targeting of the advert to the correct
recipient. Traditionally, this would be done by geographical area or
age group. More recently however, with the ability to better express
to the world your other preferences and personal situation through
social media networks, it is possible to advertise to people of a
specific relationship status, group affiliation or fans of similar
services. Facebook and Spotify are prime examples of how specific
demographic, geographic and chronological conditions are set to not
only able to return high conversion rates, and thus, higher-priced
advertising to customers, but also, as a result, able to vend smaller
advertising exposures as a tangible product. This means a greater
number of clients, and thus a more robust business model.

In a cars-as-a-service model, two potential avenues of value creation
are possible. Firstly, information on the journeys of car hire
individuals alongside the time they drive and their personal
characteristics, could be vended in a data package to companies in
retail and leisure industries. These companies would have otherwise
had to undertake expensive customer research, thus the value is
clear. However, this is very likely to suffer from extremely low
consent rates, as it would perhaps be the most invasive form of
personal data harvesting currently in existence. Alternatively,
consent rates may well be much higher if the data was instead used to
form targeted advertising at point of booking. This way, organizations
could offer discounts to individuals it feels it may be able to
convert to using their business on the trip, incentivising them to
consent to the scheme. Furthermore, this could be done dynamically
through the common advanced display systems that are available in
modern automobiles. The revenue stream here will be twofold: the
advertising organizations will pay for the in-car advertising rights,
and individuals would be more inclined to travel through the MaaS
model if special offers would be available as a user.


\subsubsection{Demand-based Pricing -- Smart/MaaS/EV}

An alternate method to control congestion is to bring economic forces
to bare on when an individual chooses to travel. In practice, a MaaS
could include an additional influence based on the expected congestion
of the roads at point of travel, attempting to deter travel that would
exacerbate the congestion. Ultimately, this requires the highest
critical resolution of hire cars of all the use cases addressed
here. Furthermore, many ethical dilemmas exist. It might be extremely
unpopular that the most sustainable cars are essentially `taxed' into
staying off the roads, while the unsustainable private transport is
free to do as it wishes.

However another interpretation is around the ability to better control
demand for the service, a key consideration for electric vehicles due
to the fact that, even in the increasing affordability of fast
chargers, EVs require considerably longer than ICE cars to transition
from zero to full range capacity.
 
Thirdly, this can also be used to manage car supply and demand between
different nodes of car collection. For example, the following
situations within a booking could threaten the ability to service
later bookings:

\begin{enumerate}
\item {\emph{Inclement driving conditions:}} EVs are susceptible to have
  sizeable variation of energy use per mile depending on driving
  conditions. Cold weather can have negative effects on electric
  torque. Additionally, heaters in EVs are unable to use the waste
  heat that a combustion car generates, so additional power from the
  battery is required; as much as 15\% in certain circumstances
  [19]. Live battery data can allow the car hire management system to
  know the exact power use of a journey.
\item {\emph{Congestion:}} Traffic can significantly decrease the efficiency of
  the EV, although the effect is less than compared to a combustion
  car as EV engine can turn off and on seamlessly. More significantly,
  congestion greatly increases the journey time. Car speed data and
  GPS location data can inform the booking management system when a
  car is stuck in traffic.
\item {\emph{Satellite navigation:}} If an individual decides to take a longer
  route home, or even a route of the same duration of time but a
  greater use of charge (such as a longer motorway route compared to a
  slower trunk route). Live satellite navigation data can inform the
  booking management system the minute the drive has this intention.
\end{enumerate}

Pre-warning of any of these unforeseen circumstances can allow
immediate shifting of the booking system to reflect increased hire
duration or increased charging duration. This will prevent people
booking in a time when the car will now be driving/charging. The value
from this system comes from improved reliability of the car hire
service.

A simulation exploring the potential impact of demand-based pricing on
revenue and variance of booking density, for a designed node of an
smart EV MaaS node, can be seen in
Figures~\ref{fig:smartpricingformula} and \ref{fig:smartpricegraphs}.

\begin{figure}[!h]
\centering
\includegraphics[width=\columnwidth]{images/smartpricingformula.png}
\caption{Smart pricing formula}
\label{fig:smartpricingformula}
\end{figure}

\begin{figure*}[!htb]
\centering
\includegraphics[width=\textwidth]{images/smartpricegraphs.png}
\caption{From left to right: Price ranges with different formula
  values; Varying total revenue under different schemes; Variance in
  booking density, a measure of the difficult of servicing bookings,
  with different ‘smart powers’ and average prices. Draws on data
  regarding price sensitivity, customer flexibility and estimated
  footfall collected during the Bristol Smart EV Hub project. (Cooper,
  Elder, Noble, Tingle, \& Wilde, 2015)}
\label{fig:smartpricegraphs}
\end{figure*}

As can be observed, it is possible to create a scenario where both
total revenue is increased and bookings are more evenly
distributed. This mechanism could be made stronger -- or seen to be
more of a `carrot' than a `stick' strategy by the use of reduced
parking costs in non- or limited-nodal MaaS systems.

\subsubsection{Driving Styles and Usage Habits -- Smart/EV/MaaS} 

As a high growth market, the EV market is currently undergoing heavy
R\&D investment. As a distinctly different driving experience,
automotive designers are particularly interested in how users interact
with the vehicle. Such data is not currently commonplace and as such;
to create this car manufacturers spend capital on customer surveys,
on-road testing and other investments.

A similar situation can be observed in the in the car insurance
industry. At present relatively few car insurers offer cover for EVs
due to a poor understanding of their risk (Cohen, 2015). Those that do
offer prices above the ICE equivalent average 16-26\% more (Cohen,
2015). It is unclear if EV owners drive in an identical manner to
combustion car owners, or if they are at a higher risk of collision
due to lower noise and potential higher accelerations than combustion
cars. Furthermore, many fundamental components of the EV have yet to
come close to their expected end of life, so it is risky for insurers
who do not know if the vehicles will reach their rated life.

Similar to the understanding of how individuals operate electric cars,
the UK’s National Grid\footnote{\url{http://www.nationalgrid.com/uk/}}
would be interested in how individuals charge their EVs, as this will
heavily influence how the grid develops in the next 50 years. Although
typically our EV’s have predetermined charging patterns, EVs that are
taken on multi-day rentals will most likely require charging to
address the specific needs of the hirer. This provides a valid insight
into the charging habits of EV users. The value from this system comes
from the created market of vending charging habits to the UK’s
National Grid. It is reasonable to assume there is value in these data
sets to the operator of an EV MaaS system, through external sale, if a
data strategy was in place to collect and store the insight.

\subsubsection{Dynamic Traffic Routing -- Smart/MaaS} 

Understanding the average speed of a road’s cars allows the city's
transport management to predict the related areas of congestion. The
necessary resolution of this is much lower i.e. the speed of a given
stretch of road is largely similar for all vehicles driving along it. 

Knowing such information in real time, the city transport management
can employ data-based traffic management techniques. One of the most
common of these is the notion of dynamic traffic rerouting. Road users
are directed to the fastest route to a given location by being mindful
of congestion. The car hire scheme takes this concept one step
further, as it will be possible to understand where individuals are
planning on driving in advance. With this information, mitigation
actions that would previously be considered to have too long a lead
time even in the `live' mode can be implemented. For example, higher
use of contraflow lanes that can be dynamically adjusted to allow for
the particular nature of the rush hour traffic, increased public
transport frequencies  to help move demand off the roads, or variable
speed limits to maximize flow rates and relieve bottlenecks.

Much of the infrastructure necessary to facilitate this has already
been tested at scale: dynamic lane direction has proven successful in
Birmingham and the M25 has extensive supporting evidence about its
dynamic speed limit interventions. This infrastructure could be
improved in effectiveness from data that was faster and more accurate
than existing, predominantly analogue sensing techniques.  A criticism
of this concept might be that such data sets are currently collected
by digital firms such as Google. Such data however is not readily
available to cities, and when it does, it typically comes with a
substantial pricetag. A MaaS service would have access to this data
and could provide it to the city’s transport team.


\subsubsection{Grid Balancing -- EV/Smart/MaaS} 

Grid balancing is a generalized term for the concept of taking action
to mitigate for surpluses or shortages in a region or nation’s
electricity generation. In some definitions, balancing involves
transferring power into the grid, but many techniques of simply
avoiding drawing power (aka. `shedding') are considered balancing by
the UK government (Change \& ofgem, 2014). The challenge of balancing
the network is becoming much larger with more distributed generation
and larger power demands. Many stakeholders in the UK’s energy
industry are increasingly providing financial incentives for these
services and view it as an important challenge of the future.

This is increasingly relevant to EVs as they represent a growing load
on the grid; some studies have suggested even moderate uptake of EVs
might increase national power demand by up to 23\% in some areas by
2021 (Barnard, 2016). However, of greater concern is how EVs will
increase the variation (essentially the peakiness) in grid demand, a
quality that is harder to service than total demand.

However, at the same time their potential to be used as charge storing
devices enables a situation whereby rather than being a burden on the
grid which needs to be minimized, they can be a positive asset to
provide charge to meet demand at peak times. Typically batteries have
a very quick response rate to grid requirements, something other
balancing solutions can lack.

This would be exacerbated in a MaaS offering, whereby the value of the
charge to the grid at that time is factored into the cost of someone
instead using that charge to drive the car. The cars could then be
used as grid balancing charge storage in periods of lower demand.


% \section{Barriers and Enablers}\label{barriersenablers}

% \subsection{Overview}

% Throughout the workshop and interview process it became apparent that
% there were substantial barriers to both the realisation of these use
% cases, and barriers to being more certain about the specific value
% they could bring. On balance, there were also certain common enablers,
% which added clarity to these two aspects.  This section profiles those
% that were communally considered to be the most important across all
% use cases.

% \subsection{Devolution}

% In the UK devolution is a defining trend of local governance. Cities
% are increasingly being given more power and more responsibility to
% fund and deliver many of their own services. Combined with a landscape
% of fiscal austerity, this combines to provide the motivation, and
% capability, to experiment with new solutions that can deliver better
% outcomes on a smaller budget, or that drive growth, sustainably,
% growing the overall city purse. Driving potential new transport
% initiatives, such as the use cases explored herein, could be such an
% action.

% \subsection{Legislation}

% At present legislation around EVs is rapidly evolving. Only recently
% in the UK it became legal to charge for electricity partiers who
% weren’t registered electricity supplier. Up until this point, although
% designed to enforce strict license requirements, restrict `cowboy'
% generators and limit electricity theft, the law greatly prohibited the
% ability to charge for electrical charging for EVs. 

% Furthermore, subsidies are a high risk reliance. Payouts for balancing
% services are complicated and have questionable longevity.  There is
% little immediate governmental impetus for demand-based energy pricing
% to support the Grid Balancing use case. However, many energy companies
% have begun distributing smart meters, which many have suggested is the
% best precursor to new demand-based tariffs.

% \subsection{Perception}

% Despite considerable feature in literature, conference and
% international economics as a whole, the value of smart technologies as
% a concept has yet to be universally realised by the business world as
% a whole. Many surveys have suggested that concepts such as Big Data
% analytics remain the domain of large corporations, and are sometimes
% viewed with risk and pessimism by many smaller, regional
% firms. Although motivation for smart city approaches is high in local
% government, a lack of private-sector buy-in becomes a prohibitive
% barrier.  The legacy of many unexpanded `demonstrator' Smart City
% projects compounds this issue (Arqiva, 2015).

% \subsection{Citizen Culture Shock}

% The freedom to use a car when one desires is heavily ingrained in
% Western society, and it's likely some of the use cases examined
% herein, particularly those that involve changing the price of using a
% car when a large number of individuals want to use one, would receive
% significant citizen backlash and so become politically unpopular. The
% most effective implementation of these is likely to be focused around
% shaping these perceptions, such as ostensibly offering financial
% advantages for `good' behavior, rather than exclusively penalizing
% `negative' use. 

% \subsection{Critical Resolutions}

% The issue of critical resolution presents an issue particularly for
% the more macro concepts.  Modelling undertaken the Smart Electric
% Transport Hub project illustrates this barrier. Using the estimated
% number of cars in a rough 2x2 mile square of urban Bristol at peak
% time, Figure 16 demonstrates a simulation of the accuracy of the
% Dynamic Traffic Routing use case, given a varying distribution of
% connected, EV cars within the car parc. The accuracy is defined as the
% ability to state whether a 10 yard x 10 yard area of the city has an
% above average level of car presence.

% As per Figure~\ref{fig:criticalresolution}, the results are
% particularly interesting; at present private EV rates (around
% 0.001\%), the ability to perceive areas of congestion is abysmal,
% correctly identifying congestion vs no congestion with around 5\%
% accuracy. However, by 0.14\%, the system has surpassed the odds of
% simple guessing, and by 0.7\%, the system has reached an impressive
% 90\% accuracy. 0.7\% corresponds to around 70 cars; the quantity you
% might envisage within such a sized zone in a multisite implementation
% of a single MaaS node in Bristol. It is also worth noting that the
% assumptions here are particularly pessimistic -- if spots of higher
% density (rather than simply above average) wished to be found, the
% accuracy will theoretically improve.

% \begin{figure}[!htb]
% \centering
% \includegraphics[width=\columnwidth]{images/criticalresolution.png}
% \caption{MATLAB modelling of the accuracy of extrapolating to find
%   congestion from connected EV cars in Bristol, UK (Cooper, Elder,
%   Noble, Tingle, \& Wilde, 2015)}
% \label{fig:criticalresolution}
% \end{figure}

% \subsection{Collective Ownership of Functioning Cars Achieved Through
%   MaaS}

% It was identified that some of the concepts identified are considered
% significantly feasible as a result of the nature of the MaaS asset
% ownership. In theory, a car park of `smart' cars that were privately
% owned and ICE powertrain could be sufficient to facilitate,
% theoretically, some of the concepts, such as Dynamic Car
% Routing. However, MaaS is associated as a core aspect of these use
% cases because the centralised ownership of the cars means that
% connected cars can be immediately purchased, and the data is
% considered the operators to use as desired, pending user consent,
% which in turn would be a requirement for use of the MaaS service in
% any way. The concept based on privately-owned car ownership however,
% would take longer to implement as individual's cars are gradually
% replaced with more intelligent cars (assuming they make mainstream
% markets). This organic growth may encounter issues of interoperability
% between cars and systems. Regardless, citizens would need to be
% persuaded to provide their data, and it would not be a requirement for
% the use of the car for them to do so.


% \subsection{Need for Public-Private Collaboration}

% By far the most substantial barrier identified during workshops and
% interviews was the requirement for public and private sector
% stakeholders to collaborate to implement many of the use cases,
% particularly those perceived to have the greatest value. This barrier
% is perceived to have two key caveats:

% \begin{itemize}
% \item The distribution of value created for difference stakeholders is
%   not aligned with the risk and investment each stakeholder is making
%   – many of the concepts herein would create value for the city they
%   are operated in, such as through congestion reductions, air quality
%   improvement, economic growth, that would not translate, even over a
%   long timeline to financial returns for the private sector operator
%   of a smart EV MaaS service. However, the value to the city could be
%   traced to tangible reductions in spending in various other
%   initiatives. As such, business models need to be explored whereby
%   clarity on these distributed benefits is established, and
%   appropriate financial arrangements made to motivate the private
%   sector to push ahead with the investment.
% \item Different capabilities and cultures between the two stakeholder
%   groups – in the context of collaborating on a transport project, a
%   perception was raised that private sector companies are
%   traditionally focused on profit rather than wider societal benefit,
%   but able to efficiently deliver services, think entrepreneurially,
%   and understand and meet customer requirements. At the same time
%   there was a concession that public sector organizations are
%   traditionally more cautious and slower to embrace digital change,
%   but have a more holistic assessment of transport solution and
%   societal benefits. Several case studies were offered by participants
%   to articulate this, particularly the failure of the Kutsplus Demand
%   Responsive Bus system in Finland (Westlake, 2016). Recognition of
%   these differences and creating strategies that embrace the qualities
%   of both will be essential if implementations are to be successful. 
% \end{itemize}



\section{Use Case Value Comparison}\label{usecasecomp}

\subsection{Workshops/Interviews}

It is clear there is great variation in the perception of value in the
identified use cases, and the certainty to which we understand this
value and the path to realise it. These are presented in
Figure~\ref{fig:valuegraph}. From observation of the results, three
categories can be generalised:

\begin{itemize}
\item Segment 3, those that have by far the greatest benefit but are
  also the most uncertain. Typically these require highest critical
  masses and require the considerable, cross-sector stakeholder
  buy-in. However, their potential value could be described as
  extreme. Discussion highlighted how these, typically have the most
  diverse forms of value, spanning environmental, social and economic
  value, beyond simply financial benefits. While due to the
  substantial barriers, there is little probability that this group
  will be implemented in the short term, the significant potential
  cannot be ignored.
\item Segment 2, with moderate benefit whose uncertainty is somewhat
  less and who enjoy good overall value/certainty ratio. These are
  typically mechanisms that involve the collection, management and
  external vending of data. How UK law changes with respect to data
  use, as well as how contract culture changes with respect to data
  transactions will have a great influence. Many local governments or
  devolved city regions, including Bristol, are aspiring to wider open
  data initiatives, a system where similar data sets are freely
  available (and reusable), and the data is realized through the new
  businesses that open as a result. There is an ongoing debate about
  how data monetization strategies, such as these, are compatible with
  open data philosophies.
\item Segment 1, with relatively low value but relatively high
  certainty. This group are best characterised as `operational'
  changes, and as such require relatively little collaboration across
  stakeholder groups. 
\end{itemize}

\begin{figure}[!htb]
\centering
\includegraphics[width=\columnwidth]{images/valuegraph.png}
\caption{Results from workshop exercises on assigning qualitative
value scores to use cases (note: log scale)}
\label{fig:valuegraph}
\end{figure}

Graphical approximations of these characteristics across the three
segments can be seen in Figure~\ref{fig:segmentcharacteristics}.

\begin{figure}[!htb]
\centering
\includegraphics[width=\columnwidth]{images/segmentcharacteristics.png}
\caption{Generalised relationships across identified segments}
\label{fig:segmentcharacteristics}
\end{figure}


% \subsection{Systems Dynamics}

% As a means of validation, these workshop findings have been compared
% to a systems dynamics model created to explore financial estimations
% of some of the identified concepts, per-node for the Hub
% project. While of a similar context, this model was created outside of
% this research initiative. A Sankey diagram out of this model can be
% seen in Figure~\ref{fig:sankey}.

% It can be observed that, allowing for small differences in the
% categorisation of use cases (some were dissected for easier
% modelling), definition of value and certainty, relative proportions
% match our workshop findings.
 
% \begin{figure}[!htb]
% \centering
% \includegraphics[width=\columnwidth]{images/sankey.png}
% \caption{Sankey diagram output of a systems dynamics model, exploring
%   the financial benefit of some of the identified use cases for a node
%   of the Hub. (Cooper, Elder, Noble, Tingle, \& Wilde, 2015)}
% \label{fig:sankey}
% \end{figure}



\section{Conclusion}\label{conclusion}

There is clearly synergistic value within the overlaps of the proposed
`TriOpt', confirming the original hypothesis. However it appears the
specifics of this are somewhat different to initial speculation. The
findings suggest concepts that combine MaaS offerings and digital
innovation are the main focus of value creation which was favored over
MaaS, EV and digital tri-overlaps. While we can still conclude that
the notion of the triple overlaps exist as an area to be considered,
the original equal status hypothesised is incorrect, and that EVs are
perhaps considered as a significant development that is set within the
context of two transformative opportunities of MaaS and digital
innovation. Reviewing the concepts that have emerged through our
methodology, value appears to span a wide range, but is overall
significant. Certainty is also variable but generally averages at a
lower point.

With respect to recommendations that can be inferred from the findings
within this paper:

\begin{itemize}
\item Segment 1 should be considered good operational practice for
  smart, MaaS EV services. However they are not transformative, and
  generally provide limited value, so should not be considered high
  priority focuses. 
\item Segment 2 should be considered as positive additional revenue
  streams for smart, MaaS EV services. In particular, they may be
  beneficial for improving business cases to the degree that such
  services can attract investment and be launched, so releasing the
  individual benefits of each ``Opt''. While offering good value for
  relative certainty, they are not transformative in and of themselves
  and should not be considered end goals.
\item Segment 3 can be considered long term strategic focuses that
  have transformative value, and have underlying mechanisms that span
  beyond transport. Currently highly uncertain, understanding these,
  conceptually, should be a long term, pan-city, high priority
  aspiration.
\end{itemize}

It is important to note that this paper does not set out to define the
value cases of the individual ``Opts'' themselves -- such as reduced
sunk-cost-induced car use in a MaaS model -- and is designed to be a
study of interaction of the three opportunities rather than appraisal
of each concept by itself -- of which there is extensive literature
already in existence as explored in the Introduction. These
recommendations however should be appreciated within the context of
the individual benefits of MaaS, smart cities and digital innovation
and electric vehicles.

Our results suggest that there are significantly more barriers than
enablers at play in these double and triple overlapping concepts. Of
most significance in the eyes of the participants, and of most
relevance to the highest-value segment of use cases, is the need for
public and private sector collaboration. It seems reasonable to
presume this is a barrier for `as-a-service' and digital innovation
concepts across a range of city services. As such, it is also a strong
recommendation of this paper that this barrier is examined in greater
detail, as explored in Section~\ref{future}.


\section{Future Work}\label{future}

For application beyond transport, it is recommended that the
underlying generic mechanisms at play that create value are
explored. The particular emphasis on MaaS and digital innovation
suggests `digitally-enabled innovative business models' may be the
best starting point for considering these. Taking away the transport
context, several can be observed in our research:

\begin{itemize}
\item Sharing of data to mutual benefit, eg. driving habits
\item Supporting new service delivery models that bring public
benefit, eg. dynamic car routing
\item Assistance of delivering public policy e.g. demand-based pricing
(to reduce congestion)
\end{itemize}

No research, as explored in this paper’s literature review, could be
found that defines a framework of these mechanisms and there is a
significant evidence base that private sector participation in city
digital initiatives is a regular criticism of the public sector
(Galbraith, 2016) (Martin, 2016) (Institute of Civil Engineers, 2015)
(Hilton, 2016) (Wilson, 2016) (Spector, 2016) (Hoare, 2016), thus it
can be hypothesized there is value in understanding this issue better.

Heavily related to this, digitally-enabled public and private
collaboration needs to be understood better, as it currently stands as
the major barrier. The distribution of value/risk/investment
identified as part of this, and articulated in a simplified causal
loop diagram in Figure~\ref{fig:causalloop}, need to be investigated.

\begin{figure}[!htb]
\centering
\includegraphics[width=\columnwidth]{images/causalloop.png}
\caption{A causal loop diagram articulating the investment/value
  dilemma of some tri-opt transport solutions; the dotted line and
  faded blue correspond to the absence of this second reinforcing
  loop, undermining the system when value is predominantly systematic
  benefits to the city.}
\label{fig:causalloop}
\end{figure}


% \appendices
% \section{Proof of the First Zonklar Equation}
% Appendix one text goes here.

% % you can choose not to have a title for an appendix
% % if you want by leaving the argument blank
% \section{}
% Appendix two text goes here.


% use section* for acknowledgment

% \section*{Acknowledgment}

% This work has been supported in part by Arup Group Ltd. through the
% Industrial Doctorate Centre in Systems at the University of Bristol,
% UK.


% Can use something like this to put references on a page
% by themselves when using endfloat and the captionsoff option.
\ifCLASSOPTIONcaptionsoff
  \newpage
\fi



% trigger a \newpage just before the given reference
% number - used to balance the columns on the last page
% adjust value as needed - may need to be readjusted if
% the document is modified later
%\IEEEtriggeratref{8}
% The "triggered" command can be changed if desired:
%\IEEEtriggercmd{\enlargethispage{-5in}}

% references section

% can use a bibliography generated by BibTeX as a .bbl file
% BibTeX documentation can be easily obtained at:
% http://mirror.ctan.org/biblio/bibtex/contrib/doc/
% The IEEEtran BibTeX style support page is at:
% http://www.michaelshell.org/tex/ieeetran/bibtex/
\bibliographystyle{IEEEtran}
% argument is your BibTeX string definitions and bibliography database(s)
\bibliography{IEEEabrv,ieeetii2016}
%
% <OR> manually copy in the resultant .bbl file
% set second argument of \begin to the number of references
% (used to reserve space for the reference number labels box)
% \begin{thebibliography}{1}

% \bibitem{IEEEhowto:kopka}
% H.~Kopka and P.~W. Daly, \emph{A Guide to \LaTeX}, 3rd~ed.\hskip 1em plus
%   0.5em minus 0.4em\relax Harlow, England: Addison-Wesley, 1999.

% \end{thebibliography}

% biography section
% 
% If you have an EPS/PDF photo (graphicx package needed) extra braces are
% needed around the contents of the optional argument to biography to prevent
% the LaTeX parser from getting confused when it sees the complicated
% \includegraphics command within an optional argument. (You could create
% your own custom macro containing the \includegraphics command to make things
% simpler here.)
%\begin{IEEEbiography}[{\includegraphics[width=1in,height=1.25in,clip,keepaspectratio]{mshell}}]{Michael Shell}
% or if you just want to reserve a space for a photo:

% \begin{IEEEbiography}{Michael Shell}
% Biography text here.
% \end{IEEEbiography}

% % if you will not have a photo at all:
% \begin{IEEEbiographynophoto}{John Doe}
% Biography text here.
% \end{IEEEbiographynophoto}

% insert where needed to balance the two columns on the last page with
% biographies
%\newpage

% \begin{IEEEbiographynophoto}{Jane Doe}
% Biography text here.
% \end{IEEEbiographynophoto}

% You can push biographies down or up by placing
% a \vfill before or after them. The appropriate
% use of \vfill depends on what kind of text is
% on the last page and whether or not the columns
% are being equalized.

%\vfill

% Can be used to pull up biographies so that the bottom of the last one
% is flush with the other column.
%\enlargethispage{-5in}



% that's all folks
\end{document}


